\documentclass{article}
\usepackage[utf8]{inputenc}
\usepackage[style=numeric, hyperref=true, backref=true, backend=biber]{biblatex}
\usepackage{setspace}
\onehalfspacing
\usepackage[margin=1.5in]{geometry}

\usepackage{ksty}
\usepackage{hyperref}
\usepackage{adjustbox}
\usepackage[figuresleft
]{rotating}
\title{ECON 1425 Final Paper}
\author{Kevin Bi}
\date{April 2020}

\addbibresource{citations.bib}
\begin{document}

\maketitle

\section{Introduction}
The goal of this paper is to examine power and relationship dynamics in the United States Congress by studying the bill sponsorship and cosponsorship behavior of legislators. The benefit of studying these two types of behaviors is that they are public behaviors 


In particular, the benefit of studying cosponsorship behavior is that it features direct interaction between two legislators, and is unidirectional. When a legislator cosponsors a piece of legislation, they declare public support of the legislation in advance of it being formally moved to a vote. Bill sponsors often actively seek cosponsors as a way of signalling the bill's support in Congress in order to increase it's chances of moving to a floor vote and of eventually being passed into law \cite{crs_sponsorship}. Cosponsorship can therefore be viewed as a desired commodity, and those who are more influential in congress will also be able to obtain more cosponsors. 

Moreover, since cosponsorship explicitly involves multiple parties, it can be viewed as a way to build relationships within Congress. Since the cosponsor offers something that the bill sponsor desires, and since the process of soliciting cosponsors may facilitate interactions between legislators, then it is plausible that legislators who cosponsor each other's bills would build a relationship. I attempt to study this affect by examining how the number of relationships a legislator has affects their decision to become a lobbyist after leaving politics, as effective lobbying requires good relationships in Congress. 

One final benefit of examining sponsorship and cosponsorship is that it typically happens only once per bill, and is explicitly removed from a bill. This is in contrast to voting on bills, which can occur multiple times, and these indicator only exists if the bill actually makes it to the floor, whereas cosponsorship data is available for any bill that is ever proposed, regardless of whether it is ever voted on or even moves past the committee stage. 

In order to study this, I examine open source data on bill cosponsorships and sponsorships from the 93rd Congress to the 115th Congress (1973 - 2018). I combine this with data on committee assignments from 103rd to the 115th (1993 - 2018) Congress, and use a measure of committee prestige based on work by Charles Stewart, who created an ordinal ranking of committee prestige in the Senate and House based on the frequency of leaving assignments in each committee \cite{stewart-committee-values}.


\section{Related Literature}
The literature on bill cosponsorship is relatively sparse, but there is some literature that is related to my goals in this paper and provide useful insights. \reminder


\section{Data and Methodology}
\subsection{Data} 
The data we used is obtained through a variety of sources. I connect the data for legislators using bioguide ids, which are unique identifiers assigned to all U.S. legislators. In some cases, the original data source did not have Bioguide-IDs for the legislator in question, in which case the data was linked using other identification methods, such as the older Thomas ID, matching the names automatically, and manual matching for cases where programmatic methods did not work. 

\subsubsection{Bills}
Data on bills, including their sponsors and cosponsors is obtained from ProPublica's bulk data on bills \cite{propublica}. The dataset metadata on bills proposed in the senate and the house from every legislative session back to the 93rd Congress, which started in 1973. This data includes the sponsor of the bill, as well as the list of cosponsors for the bill. For the purposes of this analysis, I only examine bill proposals in the House and the Senate, which are original proposals and have full force of law, and do not consider other types of legislation like non-binding resolutions or amendments to existing bills. 

\subsubsection{Legislators}
The data on individual U.S. legislators, including their party affiliation, is obtained from the ``@unitedstates project'' dataset on current congressional legislators from GitHub \cite{congress-legislators}. This also gives us data on each term that each legislator served. To label a particular legislator as being from a party, we have their party during each term. However, if a legislator changes parties at some point in their career, that legislator is simply labeled as being in ``multiple'' parties. Unfortunately, the data does not group a legislator in terms of which legislative sessions (congresses) they served, which is necessary for performing analysis with the bills data. Rather than attempting to manipulate the dates provided, I instead examine the legislator's sponsorship and cosponsorship behavior in each session. I list a legislator as being a member of a given chamber in a given session if they either sponsored or cosponsored at least one bill during that session in that chamber. While this may be an imperfect measure (for example, if some legislator did not cosponsor or sponsor any legislation in a given legislative session), I believe that any potential error is small enough in magnitude to not affect the overalll analysis.

To calculate the experience of a legislator in any given session, I simply calculate how many sessions the legislator has previously served in (by the above measure). One problem with this method is that it does not account for legislator experience prior to 1973, but so long as our statistical analysis controls for individual legislator fixed effects, this should not be a problem. 

\subsubsection{Committee Assignment Data}
Committee assignment and ranking data was obtained from a dataset gathered by Woon and Stewart \cite{stewart-committee-assignments}. This data is from the 103rd congress to the 115th congress, and includes all committee assignments for all individualls. I consider an individual as being assigned to a particular committee if they were assigned to that committee at any point within the session in order to avoid dealing with date issues. This means that as long as a given legislator is assigned to a particular committee at some point in the session, they will be considered members of that commmittee for the purposes of this analysis. 


The data on the committee assignments was linked with the legislators data by attempting to name match, as only the full name of the legislator was provided in the originall dataset. I first based the matching on the last name, and then if there were multiple individuals, on a fuzzy match of the first name. If this automated linking method was inconclusive at all, I manually linked the legislators with the committee assignments. 

This dataset also includes data on whether the legislator was in a leadership position. For the purposes of this paper, I classify a legislator that is a majority/minority leader, a majority/minority whip, or house speaker as being in a leadership role. 

\subsubsection{Committee Rankings}
The committee ranking data comes from a paper written by Charles Stewart III \cite{stewart-committee-values}. I manually linked the names of the committees with the names of the committees given in the committee assignment data. This data contained a ranking of the committee assignment in each house, as well as a z-score of the committee's coefficient using the Grosewart method, which calculates an implied probability of an individual giving up a particular committee assignment in favor of another. There were rankings provided by chamber, for two different eras, from the 81st-95th congress, and 96th-112th congress. I used the rankings for the latter era because it overlaps directly with the time frame that I am examining.

\subsubsection{Ideology}
Data on the ideology of legislators is based on Nominate scores, which assign ideological ratings on $[-1 , 1]$ in two different dimentions ideological dimensions. The first dimensions approximately represents economic ideology, and the second dimension approximately represents ideology on cultural and lifestyle issues. The Nominate scores were obtained from VoteView \cite{nominate}.

\subsubsection{Lobbying}
Data about the post-politics lobbying careers of legislators was scraped from the Center of Responsive Politics ``Revolving Door'' website \cite{revolving-door}. I wrote a web-scraper to scrape the names of the former legislators, as well as whether the website indicated they had become a lobbyist after their political career. Unfortunately, this means that the data only offers a binary indicator of whether they ever became a lobbyist after politics. More granular data was possible and available in image format on their website, but not easily scrapable. I emailed the lobbying researcher at the Center for Responsive Politics to try and obtain machine-readable granular data on post-politics career paths, but did not receive a response. 

\subsection{Methodology}
For many of the variables of analysis, they required some sort of aggregation in order to perform regression analysis. The construction of those measures is detailed here. 

\subsubsection{Cosponsor Ideology}
In order to examine the ideological diversity of the cosponsors of a given bill, I used two approaches. The first was relatively straightforward, and measures the proportion of cosponsors on a bill who were of the same party as the sponsor of the bills. 

The second approach uses the Nominate scores of the legislators. In order to calculate the ideological distance between any two legislators, I calculate the Euclidean distance. So for given legislators, $\vec{a}, \vec{b}$ with nominate scores $(a_1, a_2)$ and $(b_1, b_2)$, respectively, I define their ideological distance by
\begin{align*}
    d(\vec{a}, \vec{b}) = \|\vec{a} - \vec{b}\| = \sqrt{(a_1 - b_1)^2 + (a_2 - b_2)^2}
\end{align*}
However, the above distance measure only works for pairwise comparisons. In order to measure the average ideological variation in a bill, I compute the two dimensional variance of the cosponsors and the sponsor. So if the sponsors and cosponsors on the bill are $\vec{c}_1, \dots, \vec{c}_n$, the ideological variance of the cosponsors is calculated as
\begin{align*}
    \frac{1}{n} \sum_{i = 1}^n \|\vec{c}_i - \bar{\vec{c}}\|^2 \text{ where } \bar{\vec{c}} = \frac{1}{n} \sum_{i = 1}^n \vec{c}_i
\end{align*}

\subsubsection{Legislator Relations}
In order to measure the relationships between two legislators, I used the mutual cosponsorship of each other's bills. To construct the relationship measure between two legisators, there were a few criteria that such a measure should satisfy:
\begin{description}
    \item[Mutual cosponsorship] The relationship should be considered strong only if each cosponsors legislation of the other. Strongly one-sided relationships, where one legislator frequently cosponsors the legislation of another legisator but does not receive many cosponsorships in return, should not be considered strong relations. 
    \item[Weak dependence on shared time] Two legislators should not receive an insurmountably high relations score purely by virtue of having a long time in office together. However, it does make since for there to be some dependence on shared experience, but should not be a linear relationship.
    \item[Linearity] If we have legislators $a$ and $b$, who cosponsor twice as much lesgislation with each other as legislators $c$ and $d$, then the relationship between $a$ and $b$ should be considered twice as strong as the relationship between $c$ and $d$. 
\end{description}
With these criteria in mind, I construct the following measure of consponsorship relations. Let $f(a, b)$ be the number of times that legislator $a$ cosponsored a piece of $b$'s legislation. Let $N(a, b)$ be the number of congresses that $a$ and $b$ shared with each other. Then we can define the relationship score between legislators $a$ and $b$ as follows:
\begin{align*}
    s(a, b) = \sqrt{\frac{f(a, b) f(b, a)}{N(a, b)}}
\end{align*}
We can verify the above properties:
\begin{description}
    \item[Mutual cosponsorship] Observe that $\partials{s}{f(a, b)} = \frac{1}{2} \frac{\sqrt{f(b, a)}}{\sqrt{f(a, b) N}}$. This is increasing in the ratio $\frac{f(b, a)}{f(a, b)}$, so the marginal benefit of an additional cosponsorship from $a$ to $b$ is greater if $b$ has cospnosored many of $a$'s bills, which is precisely the property we wanted. Observe further that for some fixed number of total cosponsorships between two legislators, the relationship score is maximized when they have equal cosponsors between them. 
    \item[Weak dependence on shared time] Let $\bar{f}(a, b) = \frac{f(a, b)}{N(a, b)}$. We can rewrite the relationship score as
    \begin{align*}
        s(a, b) = \sqrt{N(a, b) \bar{f}(a, b) \bar{f}(b, a)}
    \end{align*} 
    Then we have that
    \begin{align*}
        \partials{s}{N(a, b)} \propto \frac{1}{\sqrt{N(a, b)}}
    \end{align*} 
    So the score is increasing in the number of shared years, but exhibits diminishing marginal returns to an additional year of shared experience. 
    \item[Linearity] It is clear that if we multiply each of $f(a, b)$ and $f(b, a)$ by some constant $c$, the overall score increases by a factor of $c$. 
\end{description}

\subsubsection{Committee Prestige}
Because in a given legislative session, legislators can be on multiple committees at one time, we need to aggregate the committee ``prestige'' of a legislator in a given session. I therefore offer four possible such aggregation measures:
\begin{description}
    \item[Committee count] This is the most straightforward, and simply counts the number of committees a given legislator is assigned to in a session. However, the problem with this measure is that it does not take into account the ranking of the committee, but only the number committees.
    \item[Minimum committee rank] This computes the minimum rank of a legislator's assigned committeees in a given congressional sesssion. Since a lower numerical rank is a more prestigous committee (rank 1 is the most prestigious). This is an accurate measure if we believe that only the most prestigious committee an individual is assigned to matters, and that more prestige is not offered by being on multiple high ranking committees. This also a purely ordinal ranking, and woudl not account for the top ranking committee being significantly more prestigious than the second rankng committee.
    \item[Maximum committee coefficient] This measure takes the maximum Grosewart coefficient (z-scored) of the legislator's assigned committee for a given legislative session. This also does not take into account multiple committees, but provides some cardinal score rather than just an ordinal ranking. 
    \item[Sum of rank reciprocals] This takes the sum of the reciprocals of the ranks of all of the committees a legislator is assigned to for a legislative session. This means that more prestigious committees disproportionately increase ranking, while also taking into account less prestigious committee assignments. This measure makes sense if we believe that the total prestige of all committees is what matters, but it is disadvantageous in that this measure is more ad hoc than the others. I considered using the sum of the coefficients, but the problem is that some of the coefficients are negative, and it does not seem like an additional assignment to a less prestigious committee should negatively affect prestige. 
\end{description}

\section{Results} 
\subsection{Legislator behavior overtime}
The first type of behavior I considered was how an individual's sponsorship and cosponsorship behavior changes as they move from a position of low prestige to a position of high prestige. In these regressions, I regress on the behavior of a legislator in each legislative session, and fixed effects regressions control for both individual legislator fixed effects and legislative session (time) fixed effects. Each data point is a specific legislator's behavior in a specific legislative session. In each of the following regressions, I regress the outcome variable against the experience of the individual, against the measures of aggregate committee value, and against whether the individual was in a party leadership position. 

For each, I include the result of a pure OLS regression against experience for reference, but I do not believe this to be an accurate gauge. Individual fixed effects must be controlled for because legislators that have managed to stay in office may exhibit particular types of legislative behavior. Furthermore, session (time) fixed effects must also be controlled for as legislative activity in general has increased over time, and time is colinear with experience. 

\subsubsection{Bills cosponsored}
Table \ref{tab:bills_cosponsored} displays the results of regressions of number of bills cospnsored by each legislator in a given session against their various measures of prestige during that session. Note that while the Min committee rank coefficient is negative, this should be interpreted as more prestigious committees result in more bill cosponsorship, as larger numerical rank corresponds to a less prestigious committee. In general, these results suggest that as a legislator's prestige increases in terms of committees, they cosponsor more, but that party leadership tends to cosponsor less. s

% Table created by stargazer v.5.2.2 by Marek Hlavac, Harvard University. E-mail: hlavac at fas.harvard.edu
% Date and time: Mon, Apr 20, 2020 - 16:04:34
\begin{table}[!htbp]
    \noindent
    \caption{Bills cosponsored} 
    \label{tab:bills_cosponsored} 
    \makebox[\textwidth]{\adjustbox{max width= 0.75\paperwidth}{%
    \begin{tabular}{@{\extracolsep{5pt}}lccccccc} 
        \\[-1.8ex]\hline 
        \hline \\[-1.8ex] 
         & \multicolumn{7}{c}{\textit{Dependent variable:}} \\ 
        \cline{2-8} 
        \\[-1.8ex] & \multicolumn{7}{c}{Number of bills cosponsored in legislative session} \\ 
        \\[-1.8ex] & \textit{OLS} & \multicolumn{6}{c}{\textit{panel}} \\ 
         & \textit{} & \multicolumn{6}{c}{\textit{linear}} \\ 
        \\[-1.8ex] & (1) & (2) & (3) & (4) & (5) & (6) & (7)\\ 
        \hline \\[-1.8ex] 
         Chamber (Senate) & $-$64.054$^{***}$ & $-$25.788$^{***}$ & $-$42.673$^{***}$ & $-$28.591$^{***}$ & $-$27.618$^{***}$ & $-$32.192$^{***}$ & $-$27.049$^{***}$ \\ 
          & (3.456) & (6.160) & (6.713) & (5.984) & (5.974) & (6.215) & (6.122) \\ 
          & & & & & & & \\ 
         Experience (\# of sessions) & 0.879$^{**}$ & $-$2.002 & $-$1.997 & $-$2.913 & $-$2.950 & $-$2.720 & $-$1.393 \\ 
          & (0.307) & (3.619) & (3.607) & (3.538) & (3.537) & (3.608) & (3.597) \\ 
          & & & & & & & \\ 
         Committee count &  &  & 7.376$^{***}$ &  &  &  &  \\ 
          &  &  & (1.185) &  &  &  &  \\ 
          & & & & & & & \\ 
         Min committee rank &  &  &  & $-$0.641$^{**}$ &  &  &  \\ 
          &  &  &  & (0.236) &  &  &  \\ 
          & & & & & & & \\ 
         Max committee coefficient &  &  &  &  & 4.503$^{**}$ &  &  \\ 
          &  &  &  &  & (1.475) &  &  \\ 
          & & & & & & & \\ 
         Committee rank reciprocals &  &  &  &  &  & 29.272$^{***}$ &  \\ 
          &  &  &  &  &  & (4.457) &  \\ 
          & & & & & & & \\ 
         Leadership &  &  &  &  &  &  & $-$81.307$^{***}$ \\ 
          &  &  &  &  &  &  & (9.403) \\ 
          & & & & & & & \\ 
         Constant & 215.971$^{***}$ &  &  &  &  &  &  \\ 
          & (1.983) &  &  &  &  &  &  \\ 
          & & & & & & & \\ 
        \hline \\[-1.8ex] 
        Fixed effects? & No & Yes & Yes & Yes & Yes & Yes & Yes \\ 
        Observations & 7,138 & 7,138 & 7,138 & 7,021 & 7,021 & 7,138 & 7,138 \\ 
        R$^{2}$ & 0.046 & 0.003 & 0.010 & 0.005 & 0.006 & 0.011 & 0.016 \\ 
        Adjusted R$^{2}$ & 0.046 & $-$0.254 & $-$0.245 & $-$0.252 & $-$0.252 & $-$0.245 & $-$0.238 \\ 
        \hline 
        \hline \\[-1.8ex] 
        \textit{Note:}  & \multicolumn{7}{r}{$^{*}$p$<$0.05; $^{**}$p$<$0.01; $^{***}$p$<$0.001} \\ 
        \end{tabular} 
    }}
  \end{table} 


\subsubsection{Bills sponsored}
The same regression as above is conducted except with the dependent variable as the number of bills a legislator sponsors in a legislative session. The results, displayed in Table \ref{tab:bills_sponsored}, are largely the same as with cosponsorship, except that more experienced legislators tend to sponsor more legislation (although non statistically significantly so).
\begin{table}[!htbp]
    \noindent
    \caption{Bills sponsored} 
    \label{tab:bills_sponsored} 
    \makebox[\textwidth]{\adjustbox{max width= 0.75\paperwidth}{%
    \begin{tabular}{@{\extracolsep{5pt}}lccccccc} 
        \\[-1.8ex]\hline 
        \hline \\[-1.8ex] 
         & \multicolumn{7}{c}{\textit{Dependent variable:}} \\ 
        \cline{2-8} 
        \\[-1.8ex] & \multicolumn{7}{c}{Number of bills sponsored in legislative session} \\ 
        \\[-1.8ex] & \textit{OLS} & \multicolumn{6}{c}{\textit{panel}} \\ 
         & \textit{} & \multicolumn{6}{c}{\textit{linear}} \\ 
        \\[-1.8ex] & (1) & (2) & (3) & (4) & (5) & (6) & (7)\\ 
        \hline \\[-1.8ex] 
         Chamber (Senate) & 17.099$^{***}$ & 15.613$^{***}$ & 12.258$^{***}$ & 15.277$^{***}$ & 15.497$^{***}$ & 13.788$^{***}$ & 15.543$^{***}$ \\ 
          & (0.418) & (0.983) & (1.069) & (0.980) & (0.975) & (0.983) & (0.983) \\ 
          & & & & & & & \\ 
         Experience (\# of sessions) & 0.603$^{***}$ & 0.230 & 0.231 & 0.015 & $-$0.062 & 0.026 & 0.264 \\ 
          & (0.037) & (0.578) & (0.575) & (0.580) & (0.577) & (0.571) & (0.577) \\ 
          & & & & & & & \\ 
         Committee count &  &  & 1.466$^{***}$ &  &  &  &  \\ 
          &  &  & (0.189) &  &  &  &  \\ 
          & & & & & & & \\ 
         Min committee rank &  &  &  & $-$0.138$^{***}$ &  &  &  \\ 
          &  &  &  & (0.039) &  &  &  \\ 
          & & & & & & & \\ 
         Max committee coefficient &  &  &  &  & 1.827$^{***}$ &  &  \\ 
          &  &  &  &  & (0.241) &  &  \\ 
          & & & & & & & \\ 
         Committee rank reciprocals &  &  &  &  &  & 8.345$^{***}$ &  \\ 
          &  &  &  &  &  & (0.705) &  \\ 
          & & & & & & & \\ 
         Leadership &  &  &  &  &  &  & $-$4.555$^{**}$ \\ 
          &  &  &  &  &  &  & (1.509) \\ 
          & & & & & & & \\ 
         Constant & 10.813$^{***}$ &  &  &  &  &  &  \\ 
          & (0.240) &  &  &  &  &  &  \\ 
          & & & & & & & \\ 
        \hline \\[-1.8ex] 
        Fixed effects? & No & Yes & Yes & Yes & Yes & Yes & Yes \\ 
        Observations & 7,138 & 7,138 & 7,138 & 7,021 & 7,021 & 7,138 & 7,138 \\ 
        R$^{2}$ & 0.254 & 0.043 & 0.053 & 0.045 & 0.053 & 0.066 & 0.044 \\ 
        Adjusted R$^{2}$ & 0.254 & $-$0.204 & $-$0.191 & $-$0.202 & $-$0.192 & $-$0.175 & $-$0.202 \\ 
        \hline 
        \hline \\[-1.8ex] 
        \textit{Note:}  & \multicolumn{7}{r}{$^{*}$p$<$0.05; $^{**}$p$<$0.01; $^{***}$p$<$0.001} \\ 
        \end{tabular} 
    }}
  \end{table} 

\subsubsection{Cosponsors per bill}
Table \ref{tab:cosponsors_per_bill} shows the results of regressing the number of cosponsors per bill against each of the prestige variables. More influential legislators should in theory be able to garner more cosponsors for their legislation, and in general, this appears to be the case. One notable exception is that being on more committees results in fewer cosponsors per bill, which may suggest that being on more committees results in ``low-tier'' committee bills diluting the number of cosponsors per bill.
\begin{table}[!htbp]
    \noindent
    \caption{Cosponsors per bill} 
    \label{tab:cosponsors_per_bill} 
    \makebox[\textwidth]{\adjustbox{max width= 0.75\paperwidth}{%
    \begin{tabular}{@{\extracolsep{5pt}}lccccccc} 
        \\[-1.8ex]\hline 
        \hline \\[-1.8ex] 
         & \multicolumn{7}{c}{\textit{Dependent variable:}} \\ 
        \cline{2-8} 
        \\[-1.8ex] & \multicolumn{7}{c}{Number of cosponsors per bill sponsored in legislative session} \\ 
        \\[-1.8ex] & \textit{OLS} & \multicolumn{6}{c}{\textit{panel}} \\ 
         & \textit{} & \multicolumn{6}{c}{\textit{linear}} \\ 
        \\[-1.8ex] & (1) & (2) & (3) & (4) & (5) & (6) & (7)\\ 
        \hline \\[-1.8ex] 
         Chamber (Senate) & $-$12.562$^{***}$ & $-$12.621$^{***}$ & $-$10.804$^{***}$ & $-$13.028$^{***}$ & $-$12.664$^{***}$ & $-$13.342$^{***}$ & $-$12.545$^{***}$ \\ 
          & (0.442) & (1.230) & (1.349) & (1.195) & (1.192) & (1.245) & (1.230) \\ 
          & & & & & & & \\ 
         Experience (\# of sessions) & 0.181$^{***}$ & 1.452$^{*}$ & 1.449$^{*}$ & 1.216 & 1.198 & 1.372 & 1.415 \\ 
          & (0.039) & (0.723) & (0.722) & (0.706) & (0.706) & (0.722) & (0.723) \\ 
          & & & & & & & \\ 
         Committee count &  &  & $-$0.790$^{**}$ &  &  &  &  \\ 
          &  &  & (0.242) &  &  &  &  \\ 
          & & & & & & & \\ 
         Min committee rank &  &  &  & $-$0.237$^{***}$ &  &  &  \\ 
          &  &  &  & (0.047) &  &  &  \\ 
          & & & & & & & \\ 
         Max committee coefficient &  &  &  &  & 1.697$^{***}$ &  &  \\ 
          &  &  &  &  & (0.296) &  &  \\ 
          & & & & & & & \\ 
         Committee rank reciprocals &  &  &  &  &  & 3.276$^{***}$ &  \\ 
          &  &  &  &  &  & (0.899) &  \\ 
          & & & & & & & \\ 
         Leadership &  &  &  &  &  &  & 4.752$^{*}$ \\ 
          &  &  &  &  &  &  & (1.895) \\ 
          & & & & & & & \\ 
         Constant & 16.546$^{***}$ &  &  &  &  &  &  \\ 
          & (0.255) &  &  &  &  &  &  \\ 
          & & & & & & & \\ 
        \hline \\[-1.8ex] 
        Fixed effects? & No & Yes & Yes & Yes & Yes & Yes & Yes \\ 
        Observations & 7,061 & 7,061 & 7,061 & 6,963 & 6,963 & 7,061 & 7,061 \\ 
        R$^{2}$ & 0.103 & 0.020 & 0.022 & 0.026 & 0.027 & 0.022 & 0.021 \\ 
        Adjusted R$^{2}$ & 0.103 & $-$0.234 & $-$0.232 & $-$0.228 & $-$0.226 & $-$0.231 & $-$0.233 \\ 
        \hline 
        \hline \\[-1.8ex] 
        \textit{Note:}  & \multicolumn{7}{r}{$^{*}$p$<$0.05; $^{**}$p$<$0.01; $^{***}$p$<$0.001} \\ 
        \end{tabular} 
    }}
  \end{table} 


\nocite{stargazer}

\pagebreak
\printbibliography

\end{document}
