\documentclass{article}
\usepackage[utf8]{inputenc}
\usepackage[style=numeric, hyperref=true, backref=true, backend=biber]{biblatex}
\usepackage{setspace}
\onehalfspacing
\usepackage[margin=1.5in]{geometry}

\setlength{\parskip}{1em}
\usepackage{ksty}
\usepackage{hyperref}
\usepackage{adjustbox}
\usepackage[figuresleft]{rotating}

\newtheorem{hypothesis}{Hypothesis}

\title{ECON 1425 Final Paper}
\author{Kevin Bi}
\date{April 2020}

\addbibresource{citations.bib}
\begin{document}

\maketitle

\section{Introduction}
The goal of this paper is to examine power and relationship dynamics in the United States Congress by studying the bill sponsorship and cosponsorship behavior of legislators. The benefit of studying these two types of behaviors is that they are public behaviors 


In particular, the benefit of studying cosponsorship behavior is that it features direct interaction between two legislators, and is unidirectional. When a legislator cosponsors a piece of legislation, they declare public support of the legislation in advance of it being formally moved to a vote. Bill sponsors often actively seek cosponsors as a way of signalling the bill's support in Congress in order to increase it's chances of moving to a floor vote and of eventually being passed into law \cite{crs_sponsorship}. Cosponsorship can therefore be viewed as a desired commodity, and those who are more influential in congress will also be able to obtain more cosponsors. 

Moreover, since cosponsorship explicitly involves multiple parties, it can be viewed as a way to build relationships within Congress. Since the cosponsor offers something that the bill sponsor desires, and since the process of soliciting cosponsors may facilitate interactions between legislators, then it is plausible that legislators who cosponsor each other's bills would build a relationship. I attempt to study this affect by examining how the number of relationships a legislator has affects their decision to become a lobbyist after leaving politics, as effective lobbying requires good relationships in Congress. 

One final benefit of examining sponsorship and cosponsorship is that it typically happens only once per bill, and is explicitly removed from a bill. This is in contrast to voting on bills, which can occur multiple times, and these indicator only exists if the bill actually makes it to the floor, whereas cosponsorship data is available for any bill that is ever proposed, regardless of whether it is ever voted on or even moves past the committee stage. 

In order to study this, I examine open source data on bill cosponsorships and sponsorships from the 93rd Congress to the 115th Congress (1973 - 2018). I combine this with data on committee assignments from 103rd to the 115th (1993 - 2018) Congress, and use a measure of committee prestige based on work by Charles Stewart, who created an ordinal ranking of committee prestige in the Senate and House based on the frequency of leaving assignments in each committee \cite{stewart-committee-values}.


\section{Related Literature}
\subsection{Bill Cosponsorship}
While bill cosponsorship is an official legislative act in Congress, it does not hold any legislative power. There is no (formal) requirement that a bill receive a certain number of cosponsors in order to advance to a vote nor does the number of cosponsors legally affect the passage of a bill. The question then arises of why legislators would cosponsor at all, and why cosponsorships would be desirable. Koger (2003) \cite{koger2003} argues that the number of cosponsors a bill has is a clear and public signal to party leadership as to the degree and diversity of support that a bill has. Since legislators do not want to advance bills to a vote that will clearly fail, cosponsorship serves as a valuable signal to determine whether a bill should advance in the legislative process. Bill sponsors would therefore pursue more cosponsors in order to increase the chances of their bill reaching the floor and being passed. Wilson and Young (1997) \cite{wilson_young1997} confirm that the more cosponsors a bill has, the better chance it has of moving through the legislative process and eventually to a floor vote.

While this is a plausible explanation for why a legislator would want cosponsors, what incentive would the same legislator have to cosponsor another legislator's bill. A simple straightforward explanation might be that the legislator has policy preferences, and they may want to help the bill advance if it agrees with their policy preferences. Another approach holds that bill cosponsorship may be a valuable signal to an external audience. Koger \cite{koger2003} points out that cosponsorship can be particularly useful to signal a position on a bill that does not reach a floor vote, and that legislators in the minority party tend to cosponsor more legislations. However, Kessler and Krehbiel (1996) \cite{kessler_krehbiel1996}, in an analysis of the 103rd Congress, finds that legislators in more electoral danger do not cosponsor more legislation, suggesting that any signalling value of cospnosorship is aimed primarily not at the electorate at large, but rather to insiders, including interest groups, lobbyists, or other legislators. 

Woon (2008) \cite{woon2008} presents a model of bill sponsorship behavior in which legislators must balance the ideological content of their bill with its probability of passage. If we assume the same is true for cosponsoring a bill, then legislators would want to cosponsor bills close to their own ideological positions, but that also have a high chance of passing. This also offers some insight into the cost of cosponsoring. Legislators may not want to cosponsor legislation that will fail, and they may not want to cosponsor legislation that is ideologically distant from the policy preferences of their constituents. 

\subsection{Post-politics lobbying}
A relatively common career for legislators who have left politics but want to stay involved in the political arena is to become a political lobbyist. The issue has garnered attention as Congresswoman Alexandria Ocasio Cortez and Senator Ted Cruz, legislators from opposite ends of the political spectrum, announced they would each pursue legislation to ban former members of Congress from becoming lobbyists afterwards \cite{vox_lobbying}. The so called ``revolving door'' of former legislators becoming lobbyists has also received substantially more academic attention in recent years.

Lazarus and McKay (2012) \cite{lazarus_mckay2012} find that universities who employed a former legislator or former congressional found significantly more success lobbying for earmarked funding. Vidal et al. (2012) \cite{vidal2012} found that among former staffers who became lobbyists, those who had more powerful connections tend to be paid more, and that paid declined as the number of legislators in Congress who a staffer worked for decreased. Lazarus, McKay, and Herbel (2016) \cite{lazarus_mckay_herbel2016} find that legislators with more seniority and in positions of power are more likely to become lobbyists. Finally, Mattozzi and Merlo (2008) \cite{mattozzi_merlo2008} develop a model in which politicians are not career politicians, but that politics is one part of an individual's career choice. Extending this analysis may suggest that actions in the political sphere can be changed to increase earnings in a post-political career.

\section{Theory and Hypotheses}
While most of the existing literature treats cosponsorship as a policy signalling mechanism on the part of the cosponsor, I explore its role as a way of building a relationship and currying favor with the bill sponsor. Observe that from the point of the bill sponsor, each additional cosponsor is strictly beneficial for signalling the bill's support to party leadership, to interest groups, or to the public. There is no cost to an additional cosponsor. However, for the cosponsor, the act of cosponsoring carries some political risk in signalling support for a bill that my rouse electoral opposition and that may fail. In particular for the cosponsor, they are attaching their support to a bill they did not write, and therefore cannot optimize the bill to match either their own ideology or the ideology of their electorate in the same way that the bill's sponsor can. 

However, such an analysis would predict that the number of cosponsors a bill receives depends only on its policy content, and that the average number of cosponsors per bill would remain consistent over time, regardless of that legislator's standing. We would also expect that legislators generally cosponsor the same amount of legislation over time, assuming their ideology remains consistent. Koger (2003) \cite{koger2003} presents an argument that more senior members may cosponsor more, because their cospnosorships are more valuable in signalling the viability of a bill. However, while this explains the demand side of bill cosponsorship, it does not explain why a more senior legislator would want to cosponsor more. More importantly, it would predict that senior legislators receive the same number of cosponsorships per bill as junior legislators do.

Instead, I propose that cosponsorships are a ``currency'' of sorts for purchasing political favors. These could take the form of reciprocal cosponsorships in the future, or in the hopes of receiving promotion to a higher ranking committee. Since more senior and powerful politicians (party leaders, high ranking committee members) have more influence over party promotions and control over the legislative agenda, this would predict that they receive more cosponsorships. This model would also predict that younger members cosponsor more because they have less starting political capital and can use cosponsorship to build up the political capital. Furthermore, older members in Congress have a shorter time horizon for the cosponsorships to pay future political dividends, and would thus have less incentive to cosponsor. However, one caveat to this model would be that if the cosponsorships of more influential members in Congress is more valuable, then those members would be able to extract more political favors for their cosponsorships. This would then raise the marginal value of a cosponsorship, and could incentivize more influential members to cosponsor more. Moreover, it may be because they cosponsor much legislation that they are able to achieve a high ranking position in the first place. This leads to the following hypotheses:

\begin{hypothesis}
    \label{hyp:cosponsors_per_bill}
    Higher prestige members of Congress will receive more cosponsors per bill than lower prestige members.
\end{hypothesis}
\begin{hypothesis}
    \label{hyp:bills_cosponsored_experience}
    The number of bills cosponsored per member decreases as they gain experience. 
\end{hypothesis}
\begin{hypothesis}
    \label{hyp:reciprocal_cosponsorship}
    Legislators who cosponsor more bills will also receive more cosponsors per bill.
\end{hypothesis}
\begin{hypothesis}
    \label{hyp:leadership_cosponsors}
    More influential legislators cosponsor more often.
\end{hypothesis}

If we further assume that the political risk of cosponsorship increases as a bill becomes more ideologically distant from a given legislator's (or their electorate's) ideology, but that the political benefits of cosponsoring are increasing in the influence of the bill sponsor, then we would expect more influential members of Congress to pull in more ideologically diverse cosponsors.

\begin{hypothesis}
    \label{hyp:cosponsor_ideology}
    The cosponsors of more influential legislators will be more ideologically diverse than cosponsors of less influential legislators. 
\end{hypothesis}

Finally, if cosponsorships are providing a good to another individual, then it may also be useful for building interpersonal relationships in Congress. If we build on previous literature \cite{lazarus_mckay_herbel2016} \cite{vidal2012} which finds that stronger connections in Congress lead to more lobbying, then we would assume those that cosponsor more and have stronger cosponsorship relationships will be more likely to become lobbyists.

\begin{hypothesis}
    \label{hyp:lobbying_relations}
    Legislators who have cosponsored more and have stronger ``cosponsorship relations'' will be more likely to become lobbyists.
\end{hypothesis}
\begin{hypothesis}
    \label{hyp:lobbying_prestige}
    Legislators who have more influence in Congress will be more likely to become lobbyists. 
\end{hypothesis}

\section{Data and Variables}
\subsection{Data} 
The data we used is obtained through a variety of sources. I connect the data for legislators using bioguide ids, which are unique identifiers assigned to all U.S. legislators. In some cases, the original data source did not have Bioguide-IDs for the legislator in question, in which case the data was linked using other identification methods, such as the older Thomas ID, matching the names automatically, and manual matching for cases where programmatic methods did not work. 

\subsubsection{Bills}
Data on bills, including their sponsors and cosponsors is obtained from ProPublica's bulk data on bills \cite{propublica}. The dataset includes metadata on bills proposed in the senate and the house from every legislative session back to the 93rd Congress, which started in 1973. This data includes the sponsor of the bill, as well as the list of cosponsors for the bill. For the purposes of this analysis, I only examine bill proposals in the House and the Senate, which are original proposals and have full force of law, and do not consider other types of legislation like non-binding resolutions or amendments to existing bills. 

\subsubsection{Legislators}
The data on individual U.S. legislators, including their party affiliation, is obtained from the ``@unitedstates project'' dataset on current congressional legislators from GitHub \cite{congress-legislators}. This also gives us data on each term that each legislator served. To label a particular legislator as being from a party, we have their party during each term. However, if a legislator changes parties at some point in their career, that legislator is simply labeled as being in ``multiple'' parties. Unfortunately, the data does not group a legislator in terms of which legislative sessions (congresses) they served, which is necessary for performing analysis with the bills data. Rather than attempting to manipulate the dates provided, I instead examine the legislator's sponsorship and cosponsorship behavior in each session. I list a legislator as being a member of a given chamber in a given session if they either sponsored or cosponsored at least one bill during that session in that chamber. While this may be an imperfect measure (for example, if some legislator did not cosponsor or sponsor any legislation in a given legislative session), I believe that any potential error is small enough in magnitude to not affect the overalll analysis.

To calculate the experience of a legislator in any given session, I simply calculate how many sessions the legislator has previously served in (by the above measure). One problem with this method is that it does not account for legislator experience prior to 1973, but so long as our statistical analysis controls for individual legislator fixed effects, this should not be a problem. 

\subsubsection{Committee Assignment Data}
Committee assignment and ranking data was obtained from a dataset gathered by Woon and Stewart \cite{stewart-committee-assignments}. This data is from the 103rd congress to the 115th congress, and includes all committee assignments for all individualls. I consider an individual as being assigned to a particular committee if they were assigned to that committee at any point within the session in order to avoid dealing with date issues. This means that as long as a given legislator is assigned to a particular committee at some point in the session, they will be considered members of that commmittee for the purposes of this analysis. 


The data on the committee assignments was linked with the legislators data by attempting to name match, as only the full name of the legislator was provided in the originall dataset. I first based the matching on the last name, and then if there were multiple individuals, on a fuzzy match of the first name. If this automated linking method was inconclusive at all, I manually linked the legislators with the committee assignments. 

This dataset also includes data on whether the legislator was in a leadership position. For the purposes of this paper, I classify a legislator that is a majority/minority leader, a majority/minority whip, or house speaker as being in a leadership role. 

\subsubsection{Committee Rankings}
The committee ranking data comes from a paper written by Charles Stewart III \cite{stewart-committee-values}. I manually linked the names of the committees with the names of the committees given in the committee assignment data. This data contained a ranking of the committee assignment in each house, as well as a z-score of the committee's coefficient using the Grosewart method, which calculates an implied probability of an individual giving up a particular committee assignment in favor of another. There were rankings provided by chamber, for two different eras, from the 81st-95th congress, and 96th-112th congress. I used the rankings for the latter era because it overlaps directly with the time frame that I am examining.

\subsubsection{Ideology}
Data on the ideology of legislators is based on Nominate scores, which assign ideological ratings on $[-1 , 1]$ in two different dimentions ideological dimensions. The first dimensions approximately represents economic ideology, and the second dimension approximately represents ideology on cultural and lifestyle issues. The Nominate scores were obtained from VoteView \cite{nominate}.

\subsubsection{Lobbying}
Data about the post-politics lobbying careers of legislators was scraped from the Center of Responsive Politics ``Revolving Door'' website \cite{revolving-door}. I wrote a web-scraper to scrape the names of the former legislators, as well as whether the website indicated they had become a lobbyist after their political career. Unfortunately, this means that the data only offers a binary indicator of whether they ever became a lobbyist after politics. More granular data was possible and available in image format on their website, but not easily scrapable. I emailed the lobbying researcher at the Center for Responsive Politics to try and obtain machine-readable granular data on post-politics career paths, but did not receive a response. 

\subsection{Variables}
For many of the variables of analysis, they required some sort of aggregation in order to perform regression analysis. The construction of those measures is detailed here. 

\subsubsection{Cosponsor Ideology}
In order to examine the ideological diversity of the cosponsors of a given bill, I used two approaches. The first was relatively straightforward, and measures the proportion of cosponsors on a bill who were of the same party as the sponsor of the bills. 

The second approach uses the Nominate scores of the legislators. In order to calculate the ideological distance between any two legislators, I calculate the Euclidean distance. So for given legislators, $\vec{a}, \vec{b}$ with nominate scores $(a_1, a_2)$ and $(b_1, b_2)$, respectively, I define their ideological distance by
\begin{align*}
    d(\vec{a}, \vec{b}) = \|\vec{a} - \vec{b}\| = \sqrt{(a_1 - b_1)^2 + (a_2 - b_2)^2}
\end{align*}
However, the above distance measure only works for pairwise comparisons. In order to measure the average ideological variation in a bill, I compute the two dimensional variance of the cosponsors and the sponsor. So if the sponsors and cosponsors on the bill are $\vec{c}_1, \dots, \vec{c}_n$, the ideological variance of the cosponsors is calculated as
\begin{align*}
    \frac{1}{n} \sum_{i = 1}^n \|\vec{c}_i - \bar{\vec{c}}\|^2 \text{ where } \bar{\vec{c}} = \frac{1}{n} \sum_{i = 1}^n \vec{c}_i
\end{align*}

\subsubsection{Legislator Relations}
In order to measure the relationships between two legislators, I used the mutual cosponsorship of each other's bills. To construct the relationship measure between two legisators, there were a few criteria that such a measure should satisfy:
\begin{description}
    \item[Mutual cosponsorship] The relationship should be considered strong only if each cosponsors legislation of the other. Strongly one-sided relationships, where one legislator frequently cosponsors the legislation of another legisator but does not receive many cosponsorships in return, should not be considered strong relations. 
    \item[Weak dependence on shared time] Two legislators should not receive an insurmountably high relations score purely by virtue of having a long time in office together. However, it does make since for there to be some dependence on shared experience, but should not be a linear relationship.
    \item[Linearity] If we have legislators $a$ and $b$, who cosponsor twice as much lesgislation with each other as legislators $c$ and $d$, then the relationship between $a$ and $b$ should be considered twice as strong as the relationship between $c$ and $d$. 
\end{description}
With these criteria in mind, I construct the following measure of consponsorship relations. Let $f(a, b)$ be the number of times that legislator $a$ cosponsored a piece of $b$'s legislation. Let $N(a, b)$ be the number of congresses that $a$ and $b$ shared with each other. Then we can define the relationship score between legislators $a$ and $b$ as follows:
\begin{align*}
    s(a, b) = \sqrt{\frac{f(a, b) f(b, a)}{N(a, b)}}
\end{align*}
We can verify the above properties:
\begin{description}
    \item[Mutual cosponsorship] Observe that $\partials{s}{f(a, b)} = \frac{1}{2} \frac{\sqrt{f(b, a)}}{\sqrt{f(a, b) N}}$. This is increasing in the ratio $\frac{f(b, a)}{f(a, b)}$, so the marginal benefit of an additional cosponsorship from $a$ to $b$ is greater if $b$ has cospnosored many of $a$'s bills, which is precisely the property we wanted. Observe further that for some fixed number of total cosponsorships between two legislators, the relationship score is maximized when they have equal cosponsorships in each direction. 
    \item[Weak dependence on shared time] Let $\bar{f}(a, b) = \frac{f(a, b)}{N(a, b)}$. We can rewrite the relationship score as
    \begin{align*}
        s(a, b) = \sqrt{N(a, b) \bar{f}(a, b) \bar{f}(b, a)}
    \end{align*} 
    Then we have that
    \begin{align*}
        \partials{s}{N(a, b)} \propto \frac{1}{\sqrt{N(a, b)}}
    \end{align*} 
    So the score is increasing in the number of shared years, but exhibits diminishing marginal returns to an additional year of shared experience. 
    \item[Linearity] It is clear that if we multiply each of $f(a, b)$ and $f(b, a)$ by some constant $c$, the overall score increases by a factor of $c$. 
\end{description}

\subsubsection{Committee Prestige}
Because in a given legislative session, legislators can be on multiple committees at one time, we need to aggregate the committee ``prestige'' of a legislator in a given session. I therefore offer four possible such aggregation measures:
\begin{description}
    \item[Committee count] This is the most straightforward, and simply counts the number of committees a given legislator is assigned to in a session. However, the problem with this measure is that it does not take into account the ranking of the committee, but only the number committees.
    \item[Minimum committee rank] This computes the minimum rank of a legislator's assigned committeees in a given congressional sesssion. Since a lower numerical rank is a more prestigous committee (rank 1 is the most prestigious). This is an accurate measure if we believe that only the most prestigious committee an individual is assigned to matters, and that more prestige is not offered by being on multiple high ranking committees. This also a purely ordinal ranking, and woudl not account for the top ranking committee being significantly more prestigious than the second rankng committee.
    \item[Maximum committee coefficient] This measure takes the maximum Grosewart coefficient (z-scored) of the legislator's assigned committee for a given legislative session. This also does not take into account multiple committees, but provides some cardinal score rather than just an ordinal ranking. 
    \item[Sum of rank reciprocals] This takes the sum of the reciprocals of the ranks of all of the committees a legislator is assigned to for a legislative session. This means that more prestigious committees disproportionately increase ranking, while also taking into account less prestigious committee assignments. This measure makes sense if we believe that the total prestige of all committees is what matters, but it is disadvantageous in that this measure is more ad hoc than the others. I considered using the sum of the coefficients, but the problem is that some of the coefficients are negative, and it does not seem like an additional assignment to a less prestigious committee should negatively affect prestige. 
\end{description}

\section{Results} 
\subsection{Legislator behavior overtime}
The first type of behavior I considered was how an individual's sponsorship and cosponsorship behavior changes as they move from a position of low prestige to a position of high prestige. In these regressions, I regress on the behavior of a legislator in each legislative session, and fixed effects regressions control for both individual legislator fixed effects and legislative session (time) fixed effects. Each data point is a specific legislator's behavior in a specific legislative session. In each of the following regressions, I regress the outcome variable against the experience of the individual, against the measures of aggregate committee value, and against whether the individual was in a party leadership position. 

For each, I include the result of a pure OLS regression against experience for reference, but I do not believe this to be an accurate gauge. Individual fixed effects must be controlled for because legislators that have managed to stay in office may exhibit particular types of legislative behavior. Furthermore, session (time) fixed effects must also be controlled for as legislative activity in general has increased over time, and time is colinear with experience. 

\subsubsection{Bills cosponsored}
The first area of analysis I consider is the number of bills cosposnored by each legislator. On average, senators cosponsored 141.9 bills per session, while House representatives cosponsored 190.9 bills per session. Table \ref{tab:bills_cosponsored} displays the results of regressions of number of bills cospnsored by each legislator in a given session against various measures of prestige during that session. Note that while the Min committee rank coefficient is negative, this should be interpreted as more prestigious committees result in more bill cosponsorship, as larger numerical rank corresponds to a less prestigious committee. In general, these results suggest that as a legislator's prestige increases in terms of committees, they cosponsor more, but that party leadership tends to cosponsor less. 

% Table created by stargazer v.5.2.2 by Marek Hlavac, Harvard University. E-mail: hlavac at fas.harvard.edu
% Date and time: Mon, Apr 20, 2020 - 16:04:34
\begin{table}[!htbp]
    \noindent
    \caption{Bills cosponsored} 
    \label{tab:bills_cosponsored} 
    \makebox[\textwidth]{\adjustbox{max width= 0.75\paperwidth}{%
    \begin{tabular}{@{\extracolsep{5pt}}lccccccc} 
        \\[-1.8ex]\hline 
        \hline \\[-1.8ex] 
         & \multicolumn{7}{c}{\textit{Dependent variable:}} \\ 
        \cline{2-8} 
        \\[-1.8ex] & \multicolumn{7}{c}{Number of bills cosponsored in legislative session} \\ 
        \\[-1.8ex] & \textit{OLS} & \multicolumn{6}{c}{\textit{panel}} \\ 
         & \textit{} & \multicolumn{6}{c}{\textit{linear}} \\ 
        \\[-1.8ex] & (1) & (2) & (3) & (4) & (5) & (6) & (7)\\ 
        \hline \\[-1.8ex] 
         Chamber (Senate) & $-$64.054$^{***}$ & $-$25.788$^{***}$ & $-$42.673$^{***}$ & $-$28.591$^{***}$ & $-$27.618$^{***}$ & $-$32.192$^{***}$ & $-$27.049$^{***}$ \\ 
          & (3.456) & (6.160) & (6.713) & (5.984) & (5.974) & (6.215) & (6.122) \\ 
          & & & & & & & \\ 
         Experience (\# of sessions) & 0.879$^{**}$ & $-$2.002 & $-$1.997 & $-$2.913 & $-$2.950 & $-$2.720 & $-$1.393 \\ 
          & (0.307) & (3.619) & (3.607) & (3.538) & (3.537) & (3.608) & (3.597) \\ 
          & & & & & & & \\ 
         Committee count &  &  & 7.376$^{***}$ &  &  &  &  \\ 
          &  &  & (1.185) &  &  &  &  \\ 
          & & & & & & & \\ 
         Min committee rank &  &  &  & $-$0.641$^{**}$ &  &  &  \\ 
          &  &  &  & (0.236) &  &  &  \\ 
          & & & & & & & \\ 
         Max committee coefficient &  &  &  &  & 4.503$^{**}$ &  &  \\ 
          &  &  &  &  & (1.475) &  &  \\ 
          & & & & & & & \\ 
         Committee rank reciprocals &  &  &  &  &  & 29.272$^{***}$ &  \\ 
          &  &  &  &  &  & (4.457) &  \\ 
          & & & & & & & \\ 
         Leadership &  &  &  &  &  &  & $-$81.307$^{***}$ \\ 
          &  &  &  &  &  &  & (9.403) \\ 
          & & & & & & & \\ 
         Constant & 215.971$^{***}$ &  &  &  &  &  &  \\ 
          & (1.983) &  &  &  &  &  &  \\ 
          & & & & & & & \\ 
        \hline \\[-1.8ex] 
        Fixed effects? & No & Yes & Yes & Yes & Yes & Yes & Yes \\ 
        Observations & 7,138 & 7,138 & 7,138 & 7,021 & 7,021 & 7,138 & 7,138 \\ 
        R$^{2}$ & 0.046 & 0.003 & 0.010 & 0.005 & 0.006 & 0.011 & 0.016 \\ 
        Adjusted R$^{2}$ & 0.046 & $-$0.254 & $-$0.245 & $-$0.252 & $-$0.252 & $-$0.245 & $-$0.238 \\ 
        \hline 
        \hline \\[-1.8ex] 
        \textit{Note:}  & \multicolumn{7}{r}{$^{*}$p$<$0.05; $^{**}$p$<$0.01; $^{***}$p$<$0.001} \\ 
        \end{tabular} 
    }}
  \end{table} 


\subsubsection{Bills sponsored}
The same regression as above is conducted except with the dependent variable as the number of bills a legislator sponsors in a legislative session. The results, displayed in Table \ref{tab:bills_sponsored}, are largely the same as with cosponsorship, except that more experienced legislators tend to sponsor more legislation (although non statistically significantly so).
\begin{table}[!htbp]
    \noindent
    \caption{Bills sponsored} 
    \label{tab:bills_sponsored} 
    \makebox[\textwidth]{\adjustbox{max width= 0.75\paperwidth}{%
    \begin{tabular}{@{\extracolsep{5pt}}lccccccc} 
        \\[-1.8ex]\hline 
        \hline \\[-1.8ex] 
         & \multicolumn{7}{c}{\textit{Dependent variable:}} \\ 
        \cline{2-8} 
        \\[-1.8ex] & \multicolumn{7}{c}{Number of bills sponsored in legislative session} \\ 
        \\[-1.8ex] & \textit{OLS} & \multicolumn{6}{c}{\textit{panel}} \\ 
         & \textit{} & \multicolumn{6}{c}{\textit{linear}} \\ 
        \\[-1.8ex] & (1) & (2) & (3) & (4) & (5) & (6) & (7)\\ 
        \hline \\[-1.8ex] 
         Chamber (Senate) & 17.099$^{***}$ & 15.613$^{***}$ & 12.258$^{***}$ & 15.277$^{***}$ & 15.497$^{***}$ & 13.788$^{***}$ & 15.543$^{***}$ \\ 
          & (0.418) & (0.983) & (1.069) & (0.980) & (0.975) & (0.983) & (0.983) \\ 
          & & & & & & & \\ 
         Experience (\# of sessions) & 0.603$^{***}$ & 0.230 & 0.231 & 0.015 & $-$0.062 & 0.026 & 0.264 \\ 
          & (0.037) & (0.578) & (0.575) & (0.580) & (0.577) & (0.571) & (0.577) \\ 
          & & & & & & & \\ 
         Committee count &  &  & 1.466$^{***}$ &  &  &  &  \\ 
          &  &  & (0.189) &  &  &  &  \\ 
          & & & & & & & \\ 
         Min committee rank &  &  &  & $-$0.138$^{***}$ &  &  &  \\ 
          &  &  &  & (0.039) &  &  &  \\ 
          & & & & & & & \\ 
         Max committee coefficient &  &  &  &  & 1.827$^{***}$ &  &  \\ 
          &  &  &  &  & (0.241) &  &  \\ 
          & & & & & & & \\ 
         Committee rank reciprocals &  &  &  &  &  & 8.345$^{***}$ &  \\ 
          &  &  &  &  &  & (0.705) &  \\ 
          & & & & & & & \\ 
         Leadership &  &  &  &  &  &  & $-$4.555$^{**}$ \\ 
          &  &  &  &  &  &  & (1.509) \\ 
          & & & & & & & \\ 
         Constant & 10.813$^{***}$ &  &  &  &  &  &  \\ 
          & (0.240) &  &  &  &  &  &  \\ 
          & & & & & & & \\ 
        \hline \\[-1.8ex] 
        Fixed effects? & No & Yes & Yes & Yes & Yes & Yes & Yes \\ 
        Observations & 7,138 & 7,138 & 7,138 & 7,021 & 7,021 & 7,138 & 7,138 \\ 
        R$^{2}$ & 0.254 & 0.043 & 0.053 & 0.045 & 0.053 & 0.066 & 0.044 \\ 
        Adjusted R$^{2}$ & 0.254 & $-$0.204 & $-$0.191 & $-$0.202 & $-$0.192 & $-$0.175 & $-$0.202 \\ 
        \hline 
        \hline \\[-1.8ex] 
        \textit{Note:}  & \multicolumn{7}{r}{$^{*}$p$<$0.05; $^{**}$p$<$0.01; $^{***}$p$<$0.001} \\ 
        \end{tabular} 
    }}
  \end{table} 

\subsubsection{Cosponsors per bill}
Table \ref{tab:cosponsors_per_bill} shows the results of regressing the number of cosponsors per bill against each of the prestige variables. More influential legislators should in theory be able to garner more cosponsors for their legislation, and in general, this appears to be the case. One notable exception is that being on more committees results in fewer cosponsors per bill, which may suggest that being on more committees results in ``low-tier'' committee bills diluting the number of cosponsors per bill.
\begin{table}[!htbp]
    \noindent
    \caption{Cosponsors per bill} 
    \label{tab:cosponsors_per_bill} 
    \makebox[\textwidth]{\adjustbox{max width= 0.75\paperwidth}{%
    \begin{tabular}{@{\extracolsep{5pt}}lccccccc} 
        \\[-1.8ex]\hline 
        \hline \\[-1.8ex] 
         & \multicolumn{7}{c}{\textit{Dependent variable:}} \\ 
        \cline{2-8} 
        \\[-1.8ex] & \multicolumn{7}{c}{Number of cosponsors per bill sponsored in legislative session} \\ 
        \\[-1.8ex] & \textit{OLS} & \multicolumn{6}{c}{\textit{panel}} \\ 
         & \textit{} & \multicolumn{6}{c}{\textit{linear}} \\ 
        \\[-1.8ex] & (1) & (2) & (3) & (4) & (5) & (6) & (7)\\ 
        \hline \\[-1.8ex] 
         Chamber (Senate) & $-$12.562$^{***}$ & $-$12.621$^{***}$ & $-$10.804$^{***}$ & $-$13.028$^{***}$ & $-$12.664$^{***}$ & $-$13.342$^{***}$ & $-$12.545$^{***}$ \\ 
          & (0.442) & (1.230) & (1.349) & (1.195) & (1.192) & (1.245) & (1.230) \\ 
          & & & & & & & \\ 
         Experience (\# of sessions) & 0.181$^{***}$ & 1.452$^{*}$ & 1.449$^{*}$ & 1.216 & 1.198 & 1.372 & 1.415 \\ 
          & (0.039) & (0.723) & (0.722) & (0.706) & (0.706) & (0.722) & (0.723) \\ 
          & & & & & & & \\ 
         Committee count &  &  & $-$0.790$^{**}$ &  &  &  &  \\ 
          &  &  & (0.242) &  &  &  &  \\ 
          & & & & & & & \\ 
         Min committee rank &  &  &  & $-$0.237$^{***}$ &  &  &  \\ 
          &  &  &  & (0.047) &  &  &  \\ 
          & & & & & & & \\ 
         Max committee coefficient &  &  &  &  & 1.697$^{***}$ &  &  \\ 
          &  &  &  &  & (0.296) &  &  \\ 
          & & & & & & & \\ 
         Committee rank reciprocals &  &  &  &  &  & 3.276$^{***}$ &  \\ 
          &  &  &  &  &  & (0.899) &  \\ 
          & & & & & & & \\ 
         Leadership &  &  &  &  &  &  & 4.752$^{*}$ \\ 
          &  &  &  &  &  &  & (1.895) \\ 
          & & & & & & & \\ 
         Constant & 16.546$^{***}$ &  &  &  &  &  &  \\ 
          & (0.255) &  &  &  &  &  &  \\ 
          & & & & & & & \\ 
        \hline \\[-1.8ex] 
        Fixed effects? & No & Yes & Yes & Yes & Yes & Yes & Yes \\ 
        Observations & 7,061 & 7,061 & 7,061 & 6,963 & 6,963 & 7,061 & 7,061 \\ 
        R$^{2}$ & 0.103 & 0.020 & 0.022 & 0.026 & 0.027 & 0.022 & 0.021 \\ 
        Adjusted R$^{2}$ & 0.103 & $-$0.234 & $-$0.232 & $-$0.228 & $-$0.226 & $-$0.231 & $-$0.233 \\ 
        \hline 
        \hline \\[-1.8ex] 
        \textit{Note:}  & \multicolumn{7}{r}{$^{*}$p$<$0.05; $^{**}$p$<$0.01; $^{***}$p$<$0.001} \\ 
        \end{tabular} 
    }}
  \end{table} 

  \subsubsection{Reciprocal Cosponsorship}
  I perform a somewhat crude statistical test of whether or not there is reciprocal cosponsorship, that is, those who cospnosor other legislation get more cosponsorships themselves. To do so, I simply regress the number of cosponsors per bill for a given legislator against the number of bills that legislator cosponsored themselves. I control for the number of bills sponsored as a way of controlling for how active a legislator is generally. The results of that regression are shown in Table \ref{tab:reciprocal}. Indeed, I find a weak, but statistically significant, positive relationship between the number of cosponsors per bill and the number of bills cosponsored.
  \begin{table}[!htbp] \centering 
    \caption{Reciprocal cosponsorship} 
    \label{tab:reciprocal} 
  \begin{tabular}{@{\extracolsep{5pt}}lcc} 
  \\[-1.8ex]\hline 
  \hline \\[-1.8ex] 
   & \multicolumn{2}{c}{\textit{Dependent variable:}} \\ 
  \cline{2-3} 
  \\[-1.8ex] & \multicolumn{2}{c}{Cosponsors per bill} \\ 
  \\[-1.8ex] & (1) & (2)\\ 
  \hline \\[-1.8ex] 
   Chamber (Senate) & $-$10.807$^{***}$ & $-$9.648$^{***}$ \\ 
    & (1.254) & (1.351) \\ 
    & & \\ 
   Bills cosponsored & 0.013$^{***}$ & 0.014$^{***}$ \\ 
    & (0.003) & (0.003) \\ 
    & & \\ 
   Bills sponsored & $-$0.109$^{***}$ & $-$0.113$^{***}$ \\ 
    & (0.017) & (0.017) \\ 
    & & \\ 
   Experience &  & 1.238 \\ 
    &  & (0.702) \\ 
    & & \\ 
   Leadership &  & $-$0.617 \\ 
    &  & (2.245) \\ 
    & & \\ 
   Committee count &  & $-$0.314 \\ 
    &  & (0.308) \\ 
    & & \\ 
   Committee rank reciprocals &  & $-$0.736 \\ 
    &  & (2.770) \\ 
    & & \\ 
   Min committee rank &  & 0.017 \\ 
    &  & (0.097) \\ 
    & & \\ 
   Max committee coefficient &  & 2.093 \\ 
    &  & (1.212) \\ 
    & & \\ 
  \hline \\[-1.8ex] 
  Fixed effects? & Yes & Yes \\ 
  Observations & 7,061 & 6,963 \\ 
  R$^{2}$ & 0.028 & 0.038 \\ 
  Adjusted R$^{2}$ & $-$0.223 & $-$0.213 \\ 
  \hline 
  \hline \\[-1.8ex] 
  \textit{Note:}  & \multicolumn{2}{r}{$^{*}$p$<$0.05; $^{**}$p$<$0.01; $^{***}$p$<$0.001} \\ 
  \end{tabular} 
  \end{table} 

  \subsection{Bill level analysis}
  While the previous section analyzed behaviors at a per-session individual level, we can also do an analysis at the per-bill level. However, this means that some of the data will need to be aggregated into a per bill form. To do so, I consider the most prestigious cosponsor (according to the various measures defined), or whether there exists a cosponsor with a leadership position. 

  \subsubsection{Bill enactment}
  At a very basic level, we can examine how the number and composition of cosponsors affects the likelihood that a bill is enacted into law. Unfortunately, the only data that I was able to obtain was whether the bill was eventually enacted, and not more granular data, such as whether the bill was moved out of committee or the eventual vote count of the bill. Nonetheless, the enactment data is instructive. I first conduct a basic analysis of the the probability of a bill being enacted against the number of cosponsors, the proportion of cosponsors in the sponsor's party, and the ideological variance of cosponsors on the bill. The results are shown in Table \ref{tab:enactment_simple}, and largely confirms intuitions. Bills with more cosponsors, greater proportion of cosponsors not from the sponsor's party, and higher ideological variance have a higher likelihood of passing.

  \begin{table}[!htbp] \centering 
    \caption{Bill enactment probability} 
    \label{tab:enactment_simple} 
  \begin{tabular}{@{\extracolsep{5pt}}lcccc} 
  \\[-1.8ex]\hline 
  \hline \\[-1.8ex] 
   & \multicolumn{4}{c}{\textit{Dependent variable:}} \\ 
  \cline{2-5} 
  \\[-1.8ex] & \multicolumn{4}{c}{Bill enacted} \\ 
  \\[-1.8ex] & (1) & (2) & (3) & (4)\\ 
  \hline \\[-1.8ex] 
   Chamber (Senate) & $-$0.003 & $-$0.014$^{***}$ & $-$0.003$^{*}$ & $-$0.006$^{***}$ \\ 
    & (0.002) & (0.002) & (0.002) & (0.002) \\ 
    & & & & \\ 
   Total cosponsors & 0.0003$^{***}$ &  &  & 0.0002$^{***}$ \\ 
    & (0.00002) &  &  & (0.00002) \\ 
    & & & & \\ 
   Same party cosponsors proportion &  & $-$0.068$^{***}$ &  & $-$0.046$^{***}$ \\ 
    &  & (0.002) &  & (0.003) \\ 
    & & & & \\ 
   Nominate variance &  &  & 0.187$^{***}$ & 0.108$^{***}$ \\ 
    &  &  & (0.006) & (0.007) \\ 
    & & & & \\ 
   Constant & 0.044$^{***}$ & 0.097$^{***}$ & $-$0.018$^{***}$ & 0.038$^{***}$ \\ 
    & (0.003) & (0.003) & (0.004) & (0.005) \\ 
    & & & & \\ 
  \hline \\[-1.8ex] 
  Legislative session fixed effects? & Yes & Yes & Yes & Yes \\ 
  Observations & 84,256 & 84,256 & 84,156 & 84,156 \\ 
  R$^{2}$ & 0.004 & 0.015 & 0.015 & 0.020 \\ 
  Adjusted R$^{2}$ & 0.004 & 0.015 & 0.015 & 0.019 \\ 
  \hline 
  \hline \\[-1.8ex] 
  \textit{Note:}  & \multicolumn{4}{r}{$^{*}$p$<$0.05; $^{**}$p$<$0.01; $^{***}$p$<$0.001} \\ 
  \end{tabular} 
  \end{table} 

  \subsubsection{Ideological diversity}
  The legislator level regressions on the number of cosponsors per bill already tell us that prestige has a positive impact on the number of total cosponsors. However, we can also get an idea for how influential committee assignmets and party leadership are by examining their ability to rally legislators of different ideologies.
  
  First, I examine how the various prestige measures affects the proportion of cosponsors who are in the same party as the sponsor. These results are shown in Table \ref{tab:party_prop}. Experience seems to correlate with becoming more bipartisan. Party leaders also have a higher proportion of same party cosponsors, indicating that they may have more sway in their own party but less infuence with members of other parties. The committee data all seem to suggest that as an individual gains more prestige, they are better able to bring cosponsors from the opposing party. 

  \begin{table}[!htbp] \centering 
    \caption{Same party cosponsorship proportion by sponsor traits} 
    \label{tab:party_prop} 
    \makebox[\textwidth]{\adjustbox{max width= 0.75\paperwidth}{%
  \begin{tabular}{@{\extracolsep{5pt}}lccccc} 
  \\[-1.8ex]\hline 
  \hline \\[-1.8ex] 
   & \multicolumn{5}{c}{\textit{Dependent variable:}} \\ 
  \cline{2-6} 
  \\[-1.8ex] & \multicolumn{5}{c}{Same party cosponsorship proportion} \\ 
  \\[-1.8ex] & (1) & (2) & (3) & (4) & (5)\\ 
  \hline \\[-1.8ex] 
   Chamber (Senate) & $-$0.108$^{***}$ & $-$0.112$^{***}$ & $-$0.089$^{***}$ & $-$0.098$^{***}$ & $-$0.091$^{***}$ \\ 
    & (0.003) & (0.003) & (0.004) & (0.003) & (0.003) \\ 
    & & & & & \\ 
   Experience & $-$0.002$^{***}$ & $-$0.002$^{***}$ & $-$0.002$^{***}$ & $-$0.001$^{***}$ & $-$0.001$^{*}$ \\ 
    & (0.0003) & (0.0003) & (0.0003) & (0.0003) & (0.0003) \\ 
    & & & & & \\ 
   Leadership &  & 0.130$^{***}$ &  &  &  \\ 
    &  & (0.009) &  &  &  \\ 
    & & & & & \\ 
   Committee count &  &  & $-$0.009$^{***}$ &  &  \\ 
    &  &  & (0.001) &  &  \\ 
    & & & & & \\ 
   Minimum committee rank &  &  &  & 0.003$^{***}$ &  \\ 
    &  &  &  & (0.0002) &  \\ 
    & & & & & \\ 
   Committee rank reciprocals &  &  &  &  & $-$0.054$^{***}$ \\ 
    &  &  &  &  & (0.004) \\ 
    & & & & & \\ 
   Constant & 0.701$^{***}$ & 0.702$^{***}$ & 0.720$^{***}$ & 0.677$^{***}$ & 0.716$^{***}$ \\ 
    & (0.005) & (0.005) & (0.006) & (0.006) & (0.005) \\ 
    & & & & & \\ 
  \hline \\[-1.8ex] 
  Legislative session fixed effects? & Yes & Yes & Yes & Yes & Yes \\ 
  Observations & 84,256 & 84,256 & 84,256 & 83,722 & 84,256 \\ 
  R$^{2}$ & 0.032 & 0.034 & 0.032 & 0.033 & 0.034 \\ 
  Adjusted R$^{2}$ & 0.031 & 0.034 & 0.032 & 0.033 & 0.034 \\ 
  \hline 
  \hline \\[-1.8ex] 
  \textit{Note:}  & \multicolumn{5}{r}{$^{*}$p$<$0.05; $^{**}$p$<$0.01; $^{***}$p$<$0.001} \\ 
  \end{tabular} 
    }}
  \end{table} 

  However, party affiliation alone gives us limited granularity in the data. We can also examine the ideological variance of the bill based on the various prestige measures. The results are shown in Table \ref{tab:nominate_variance}, and importantly, four of the regressions control for the proportion of cosponsors in the same party as the sponsor. This helps us gauge if some members, even if they are not as effective in bringing cosponsors from other parties, are at least more effective at courting cosponsors of varying ideologies from within in the same party.
  
  For the most part, committee level prestige seems to have the same effect within party and across parties, but notably, party leadership seems to be more effective at increasing ideological diversity of cospnosors only within the party, but not across party lines. The sponsor's experience also does not seem to play a statistically significant role in the diversity of the bill's cosponsors. 

  \begin{sidewaystable}[!htbp] \centering 
    \caption{Nominate variance by sponsor traits} 
    \label{tab:nominate_variance} 
    \makebox[\textwidth]{\adjustbox{max width= 0.75\paperheight}{%
    \begin{tabular}{@{\extracolsep{5pt}}lccccccccc} 
        \\[-1.8ex]\hline 
        \hline \\[-1.8ex] 
         & \multicolumn{9}{c}{\textit{Dependent variable:}} \\ 
        \cline{2-10} 
        \\[-1.8ex] & \multicolumn{9}{c}{Nominate variance} \\ 
        \\[-1.8ex] & (1) & (2) & (3) & (4) & (5) & (6) & (7) & (8) & (9)\\ 
        \hline \\[-1.8ex] 
         Chamber (Senate) & $-$0.019$^{***}$ & $-$0.019$^{***}$ & $-$0.017$^{***}$ & $-$0.023$^{***}$ & $-$0.025$^{***}$ & $-$0.043$^{***}$ & $-$0.036$^{***}$ & $-$0.044$^{***}$ & $-$0.044$^{***}$ \\ 
          & (0.001) & (0.001) & (0.001) & (0.001) & (0.001) & (0.001) & (0.001) & (0.001) & (0.001) \\ 
          & & & & & & & & & \\ 
         Experience & $-$0.0001 & $-$0.0001 & $-$0.0001 & $-$0.0003$^{***}$ & $-$0.0004$^{***}$ & $-$0.0005$^{***}$ & $-$0.001$^{***}$ & $-$0.001$^{***}$ & $-$0.001$^{***}$ \\ 
          & (0.0001) & (0.0001) & (0.0001) & (0.0001) & (0.0001) & (0.0001) & (0.0001) & (0.0001) & (0.0001) \\ 
          & & & & & & & & & \\ 
         Same party cosponsors proportion &  &  &  &  &  & $-$0.212$^{***}$ & $-$0.212$^{***}$ & $-$0.212$^{***}$ & $-$0.212$^{***}$ \\ 
          &  &  &  &  &  & (0.001) & (0.001) & (0.001) & (0.001) \\ 
          & & & & & & & & & \\ 
         Leadership &  & $-$0.015$^{***}$ &  &  &  & 0.013$^{***}$ &  &  &  \\ 
          &  & (0.003) &  &  &  & (0.003) &  &  &  \\ 
          & & & & & & & & & \\ 
         Committee count &  &  & $-$0.001$^{**}$ &  &  &  & $-$0.003$^{***}$ &  &  \\ 
          &  &  & (0.0005) &  &  &  & (0.0004) &  &  \\ 
          & & & & & & & & & \\ 
         Minimum committee rank &  &  &  & $-$0.001$^{***}$ &  &  &  & $-$0.0004$^{***}$ &  \\ 
          &  &  &  & (0.0001) &  &  &  & (0.0001) &  \\ 
          & & & & & & & & & \\ 
         Committee rank reciprocals &  &  &  &  & 0.018$^{***}$ &  &  &  & 0.006$^{***}$ \\ 
          &  &  &  &  & (0.001) &  &  &  & (0.001) \\ 
          & & & & & & & & & \\ 
         Constant & 0.361$^{***}$ & 0.361$^{***}$ & 0.364$^{***}$ & 0.370$^{***}$ & 0.356$^{***}$ & 0.510$^{***}$ & 0.517$^{***}$ & 0.513$^{***}$ & 0.508$^{***}$ \\ 
          & (0.002) & (0.002) & (0.002) & (0.002) & (0.002) & (0.002) & (0.002) & (0.002) & (0.002) \\ 
          & & & & & & & & & \\ 
        \hline \\[-1.8ex] 
        Legislative session fixed effects? & Yes & Yes & Yes & Yes & Yes & Yes & Yes & Yes & Yes \\ 
        Observations & 84,156 & 84,156 & 84,156 & 83,622 & 84,156 & 84,156 & 84,156 & 83,622 & 84,156 \\ 
        R$^{2}$ & 0.028 & 0.029 & 0.028 & 0.030 & 0.030 & 0.354 & 0.354 & 0.353 & 0.354 \\ 
        Adjusted R$^{2}$ & 0.028 & 0.028 & 0.028 & 0.030 & 0.030 & 0.354 & 0.354 & 0.353 & 0.354 \\ 
        \hline 
        \hline \\[-1.8ex] 
        \textit{Note:}  & \multicolumn{9}{r}{$^{*}$p$<$0.05; $^{**}$p$<$0.01; $^{***}$p$<$0.001} \\ 
        \end{tabular} 
    }}
  \end{sidewaystable} 


  \subsection{Lobbying} 
  In this section, I examine the probability that a given legislator becomes a lobbyist after they leave congress. Throughout, I control for party and retirement session fixed effects by regressing on indicators of both the party and the last congressioanl session where the legislator was active. All measures of prestige and participation (leadership, committee assignments) are taken for the last session in which the legislator was in congress.

  \subsubsection{Basic insights}
  I first examine some fairly straightforward insights into who becomes a lobbyist, such as the party of the individual, such as their party, chamber, experience, and time since the legislator's last session. Note that the number of sessions since leaving politics is calculated as 116 (the current session) minus their last active congressional session. I also obviously omit last session fixed effects when examining the effect of time since leaving congress. 

  The results of the regression are shown in Table \ref{tab:lobbyist_basic}. The only consistently statistically significant covariates are Experience and the number of sessions since leaving. However, the number of sessions since leaving is a bit misleading, because the dataset marks a former legislator as a lobbyist if they have ever been a lobbyist post-politics. More years since leaving politics means a wider window to have been in lobbying at some point. Nonetheless, this suggests that the final legislative session is necessary to control for in the subsequent regressions.
  
  \begin{table}[!htbp] \centering 
    \caption{Probability of becoming a lobbyist} 
    \label{tab:lobbyist_basic} 
  \begin{tabular}{@{\extracolsep{5pt}}lcccccc} 
  \\[-1.8ex]\hline 
  \hline \\[-1.8ex] 
   & \multicolumn{6}{c}{\textit{Dependent variable:}} \\ 
  \cline{2-7} 
  \\[-1.8ex] & \multicolumn{6}{c}{Became lobbyist} \\ 
  \\[-1.8ex] & (1) & (2) & (3) & (4) & (5) & (6)\\ 
  \hline \\[-1.8ex] 
    Chamber (Senate) & $-$0.075 & $-$0.098$^{*}$ & $-$0.057 & $-$0.074 & $-$0.094$^{*}$ & $-$0.099 \\ 
    & (0.051) & (0.047) & (0.048) & (0.047) & (0.047) & (0.052) \\ 
    & & & & & & \\ 
   Experience &  & 0.019$^{***}$ &  &  & 0.022$^{***}$ & 0.012$^{*}$ \\ 
    &  & (0.005) &  &  & (0.005) & (0.005) \\ 
    & & & & & & \\ 
    Sessions since leaving &  &  & 0.032$^{***}$ &  & 0.036$^{***}$ &  \\ 
    &  &  & (0.004) &  & (0.004) &  \\ 
    & & & & & & \\ 
   Multiple parties &  &  &  & 0.034 & 0.079 & 0.132 \\ 
    &  &  &  & (0.110) & (0.109) & (0.119) \\ 
    & & & & & & \\ 
   Republican &  &  &  & 0.034 & 0.052 & 0.054 \\ 
    &  &  &  & (0.040) & (0.037) & (0.040) \\ 
    & & & & & & \\ 
  \hline \\[-1.8ex] 
  Last session fixed effects? & Yes & Yes & No & Yes & No & Yes \\ 
  Observations & 473 & 471 & 471 & 471 & 471 & 472 \\ 
  R$^{2}$ & 0.004 & 0.258 & 0.140 & 0.230 & 0.186 & 0.022 \\ 
  Adjusted R$^{2}$ & 0.002 & 0.218 & 0.136 & 0.187 & 0.177 & 0.014 \\ 
  \hline 
  \hline \\[-1.8ex] 
  \textit{Note:}  & \multicolumn{6}{r}{$^{*}$p$<$0.05; $^{**}$p$<$0.01; $^{***}$p$<$0.001} \\ 
  \end{tabular} 
  \end{table} 

  \subsubsection{Lobbying and cosponsorship relations}
  I now examine how a legislator's relations in congress affect their probability of becoming a lobbyist after leaving Congress. To do so, I use the relationship measure defined from before and I sum up the total relations score for each former legislator. Crucially though, I only sum individuals that are still active in the 116th Congress, because these are people who would be lobbied. I also classify any legislator with whom a given legislator has non-zero cosponsor relations score as a ``friend'', and regress on the number of remaining friends in Congress. Throughout, I control for the last active session of each former legislator, their chamber, political party, and experience. Since these covariates function as controls, I omit them from the regression table. The results are shown in \ref{tab:lobbying_relations}. The results largely conform with expectations, that as a legislator's current legislative relationships improve, they are more likely to become a lobbyist after leaving congress. 


  \begin{table}[!htbp] \centering 
    \caption{Post-politics lobbying and legisative relationships} 
    \label{tab:lobbying_relations} 
    \begin{tabular}{@{\extracolsep{5pt}}lcccc} 
        \\[-1.8ex]\hline 
        \hline \\[-1.8ex] 
         & \multicolumn{4}{c}{\textit{Dependent variable:}} \\ 
        \cline{2-5} 
        \\[-1.8ex] & \multicolumn{4}{c}{Became lobbyist} \\ 
        \\[-1.8ex] & (1) & (2) & (3) & (4)\\ 
        \hline \\[-1.8ex] 
         Current relations score & 0.001$^{*}$ &  &  &  \\ 
          & (0.0003) &  &  &  \\ 
          & & & & \\ 
         Remaining friends &  & 0.002$^{*}$ &  &  \\ 
          &  & (0.001) &  &  \\ 
          & & & & \\ 
         Bills cosponsored in last session &  &  & 0.0004$^{*}$ &  \\ 
          &  &  & (0.0002) &  \\ 
          & & & & \\ 
         Cosponsors per bill in last session &  &  &  & 0.001 \\ 
          &  &  &  & (0.001) \\ 
          & & & & \\ 
        \hline \\[-1.8ex] 
        Observations & 471 & 471 & 471 & 464 \\ 
        R$^{2}$ & 0.267 & 0.270 & 0.268 & 0.262 \\ 
        Adjusted R$^{2}$ & 0.223 & 0.225 & 0.223 & 0.216 \\ 
        \hline 
        \hline \\[-1.8ex] 
        \textit{Note:}  & \multicolumn{4}{r}{$^{*}$p$<$0.05; $^{**}$p$<$0.01; $^{***}$p$<$0.001} \\ 
        \end{tabular} 
  \end{table} 
\nocite{stargazer}

\subsubsection{Lobbying and prestige}
One final relevant area of post-politics lobbying is examining the effect of prestige while in Congress. To this end, I examine the level of prestige in the final legislative session where a former legislator was active. I once again control for party, chamber, experience, and last session effects, but omit the coefficients from display. The results in are shown in \ref{tab:lobbying_prestige}. None of the major measures of prestige proved statistically significant. 
\begin{table}[!htbp] \centering 
    \caption{Post-politics lobbying and political prestige} 
    \label{tab:lobbying_prestige} 
  \begin{tabular}{@{\extracolsep{5pt}}lccccc} 
  \\[-1.8ex]\hline 
  \hline \\[-1.8ex] 
   & \multicolumn{5}{c}{\textit{Dependent variable:}} \\ 
  \cline{2-6} 
  \\[-1.8ex] & \multicolumn{5}{c}{Became lobbyist} \\ 
  \\[-1.8ex] & (1) & (2) & (3) & (4) & (5)\\ 
  \hline \\[-1.8ex] 
   Last session leadership & $-$0.129 &  &  &  &  \\ 
    & (0.136) &  &  &  &  \\ 
    & & & & & \\ 
   Last session committee count &  & $-$0.020 &  &  &  \\ 
    &  & (0.021) &  &  &  \\ 
    & & & & & \\ 
   Last session min committee rank &  &  & $-$0.001 &  &  \\ 
    &  &  & (0.004) &  &  \\ 
    & & & & & \\ 
   Last session max committeee coefficient &  &  &  & 0.004 &  \\ 
    &  &  &  & (0.023) &  \\ 
    & & & & & \\ 
    Last session committee rank recips &  &  &  &  & 0.001 \\ 
    &  &  &  &  & (0.063) \\ 
    & & & & & \\ 
  \hline \\[-1.8ex] 
  Observations & 471 & 471 & 353 & 353 & 471 \\ 
  R$^{2}$ & 0.261 & 0.261 & 0.247 & 0.247 & 0.260 \\ 
  Adjusted R$^{2}$ & 0.216 & 0.216 & 0.209 & 0.209 & 0.215 \\ 
  \hline 
  \hline \\[-1.8ex] 
  \textit{Note:}  & \multicolumn{5}{r}{$^{*}$p$<$0.05; $^{**}$p$<$0.01; $^{***}$p$<$0.001} \\ 
  \end{tabular} 
  \end{table}


  \section{Discussion}
  \subsection{Legislative Behavior}
  \reminder
\pagebreak
\printbibliography

\end{document}
